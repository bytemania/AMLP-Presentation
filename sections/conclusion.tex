\section{Conclusion}

\begin{frame}[shrink]{Conclusion: Key Findings}

\begin{itemize}
    \item Trained on \alert{10 years of daily Workday stock data} and technical
    indicators was effective even with a small dataset.
    \item \alert{LSTM-BiGRU consistently outperformed LSTM} across all evaluation
    metrics.
    \item \alert{30-day input sequences} is good enough to
        capture momentum and seasonality patterns.
    \item Early stopping and random search tuning produced robust, generalizable models.
    \item \alert{Confirms deep learning’s strength} -- especially recurrent networks for 
        sequential stock price modeling.
\end{itemize}

\note{
Both models generalized well, but the LSTM-BiGRU model demonstrated superior forecasting ability, especially during volatile periods. 
Low dropout and well-tuned architectures were key to the success of the models.
}
\end{frame}

\begin{frame}[shrink]{Conclusion: Future Work}

\begin{itemize}
    \item Incorporate \alert{sentiment analysis} from news and social media (e.g., Bidirectional Encoder Representations from Transformers (BERT) embeddings).
    \item Explore \alert{higher-frequency data} (e.g., hourly prices) for finer-grained prediction.
    \item Expand to multiple stocks or sector-based prediction to validate model generalization.
    \item \alert{Expand from next-day (single-step) prediction to multi-step forecasting}, 
    predicting multiple future closing prices over a defined horizon.
\end{itemize}

\note{
Future extensions would enrich the predictive models further.
Sentiment analysis, higher-frequency data, and transformers are promising directions.
The study lays a solid foundation for expanding deep learning-based financial forecasting.
}
\end{frame}
